\documentclass[a4paper,12 pt]{article}
\usepackage[head=0.2cm, bottom=3cm, voffset=-20pt]{geometry}
\newcommand{\hN}{\_NAME}
\usepackage{comment}
% colours
\usepackage{xcolor}
% VS
\definecolor{VSblue}{RGB}{102, 152, 206}
\definecolor{VSblue1}{RGB}{157, 201, 230}
\definecolor{VSOrange}{RGB}{197, 148 124}
\definecolor{VSblack}{RGB}{31,31,31}
\definecolor{VSwhite}{RGB}{200,200 200}
% C
\definecolor{CGreen}{RGB}{0, 110, 95}
\definecolor{CBlue}{RGB}{0, 65, 165}
\definecolor{CRed}{RGB}{183, 141, 18}
\definecolor{CPurple}{RGB}{65, 28, 53}
\definecolor{CWhite}{RGB}{241,240, 237}
\definecolor{CWhite}{RGB}{249, 241, 219}
% bear (B)
\definecolor{Bbg}{RGB}{252, 246, 229}
\definecolor{Btitle}{RGB}{64, 79, 83}
\definecolor{Btext}{RGB}{52, 62, 68}
% title settings
\usepackage{titlesec}
\titleformat*{\section}{\bfseries \Large \color{Btitle} }
\titlelabel{\thetitle.\quad}
% pagecolor
\pagecolor{Bbg}
% body
\begin{document}
\color{Btext}
% titles
\begin{center}
	\LARGE\bfseries
	Web design notes
\end{center}
\section{Django}
\subsection{basics}
	\subsubsection{html with django}
	Use \texttt{\{\{VAR\_html\}\}} for html variables.\\
	Use \texttt{\{\% PYTHON\_COMMANDS\%\}} for python commands. Remember to end a loop with \texttt{\{\% end<LOOP\_NAME>\%\}}.
	\subsubsection{important packages}
	\ttfamily
	django.urls:
	\begin{itemize}
		\item include
		\item path('path\_name', content the path renders, name="name\_of\_the\_path")
		
		\textnormal{
		\texttt{'path\_name'}:
		The default option is \texttt{""}. It could also be a variable,for example, \texttt{"<str:VAR\_NAME>"}.
		}
	\end{itemize}
	django.shortcuts:
	\begin{itemize}
		\item render(request, "FILE\_NAME.html", \{"VAR\_html":VAR\_python\})
	\end{itemize}
	\subsection{basic setup}
	\begin{itemize}
	\ttfamily
		\item pip3 install Django
		\item dango-admin startproject PROJECT\hN
		\item python manage.py startapp APP\hN
	\end{itemize}
	\normalfont
	When creating a new app, add the app's name to the list of \texttt{INSTALLED\_APPS} in \textit{settings.py}. Also create a \textit{urls.py} file in the \textit{APP\hN} folder.
	
	To direct the url to the desired app, \texttt{include("APP\hN.urls")} in the \texttt{path 'APP\hN/'}. Put the \texttt{path} in the list of \texttt{urlpatterns}.		
	\subsection{routes}
		\begin{itemize}
		\item kk
	\end{itemize}

\end{document}