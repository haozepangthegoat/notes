\documentclass[a4paper]{article}
\usepackage[head=0.2cm, bottom=3cm, voffset=-20pt]{geometry}
\newcommand{\hN}{\_NAME}
% code listings
\usepackage{listings}
\lstset{
  %language         = C++,
  basicstyle       = \ttfamily \color{VSwhite},
  keywordstyle     = \color{VSblue}\textbf,
  %keywordstyle     = \color{VSblue1}\textbf
  commentstyle     = \color{gray},
  stringstyle      = \color{VSwhite},
  stringstyle      = \color{VSOrange},
  columns          = fullflexible,
  numbers          = left,
  numberstyle      = \scriptsize\sffamily\color{gray},
  %caption          = A hello world program in \Cpp,
  xleftmargin      = 0.16\textwidth,
  xrightmargin     = 0.16\textwidth,
  showstringspaces = false,
  float,
}
\usepackage{tcolorbox}
% colours
\usepackage{xcolor}
% VS
\definecolor{VSblue}{RGB}{102, 152, 206}
\definecolor{VSblue1}{RGB}{157, 201, 230}
\definecolor{VSOrange}{RGB}{197, 148 124}
\definecolor{VSblack}{RGB}{31,31,31}
\definecolor{VSwhite}{RGB}{200,200 200}
% C
\definecolor{CGreen}{RGB}{0, 110, 95}
\definecolor{CBlue}{RGB}{0, 65, 165}
\definecolor{CRed}{RGB}{183, 141, 18}
\definecolor{CPurple}{RGB}{65, 28, 53}
\definecolor{CWhite}{RGB}{241,240, 237}
\definecolor{CWhite}{RGB}{249, 241, 219}
% bear
\definecolor{Bbg}{RGB}{252, 246, 229}
\definecolor{Btitle}{RGB}{64, 79, 83}
\definecolor{Btext}{RGB}{52, 62, 68}
%
\usepackage{titlesec}
\titleformat*{\section}{\bfseries \Large \color{Btitle} }
\titlelabel{\thetitle.\quad}
%
\pagecolor{Bbg}
% document
\begin{document}
\color{Btext}
\begin{center}
	\LARGE\bfseries
	Web design notes
\end{center}
\section{Git}
\ttfamily
commands:
\begin{itemize}
		\item mkdir testfolder
		\item git clone REPOSITORY\_URL
		\item touch NEW\_FILE\_NAME
		\item git add FILE\_NAME
		\item git add . 
		\item git commit -m "messages"
		\item git push
		\item git commit -am "messages"
		\item git pull
		\item git branch
		\item git checkout -b NEW\_BRANCH
		\item git checkout BRANCH\_NAME\\
	
		\color{CBlue}
		\textbf{Advanced}
		\item git log
		\item git revert --hard COMMIT\_HASH
		\item git push --set-upstream origin colour\_box
\end{itemize}
\normalfont
To exclude files from git, create a file \texttt{.gitignore} and add the file intended to exclude to \texttt{.gitignore}. Use \texttt{*} as wildword. 

%S
\section{HTML}
%box1
\begin{tcolorbox}[title= \centering Example code, colback= VSblack, width =150mm]
\centering
	\begin{lstlisting}[language = HTML]
<!DOCTYPE html>
<html lang="en">
    <head>
        <title>HTML Elements</title>
    </head>
    <body>
        <!-- We can create headings using h1 through h6 as tags. -->
        <h1>A Large Heading</h1>
        <h2>A Smaller Heading</h2>
        <h6>The Smallest Heading</h6>

        <!-- The strong and i tags give us bold and italics respectively. -->
        A <strong>bold</strong> word and an <i>italicized</i> word!

        <!-- We can link to another page (such as cs50's page) using a. -->
        View the <a href="https://cs50.harvard.edu/">CS50 Website</a>!

        <!-- We used ul for an unordered list and ol for an ordered one. both ordered and unordered lists contain li, or list items. -->
        An unordered list:
        <ul>
            <li>foo</li>
            <li>bar</li>
            <li>baz</li>
        </ul>
        An ordered list:
        <ol>
            <li>foo</li>
            <li>bar</li>
            <li>baz</li>
        </ol>

        <!-- Images require a src attribute, which can be either the path to a file on your computer or the link to an image online. It also includes an alt attribute, which gives a description in case the image can't be loaded. -->
        An image:
        <img src="../../images/duck.jpeg" alt="Rubber Duck Picture">
        <!-- We can also see above that for some elements that don't contain other ones, closing tags are not necessary. -->

        <!-- Here, we use a br tag to add white space to the page. -->
        <br/> <br/>

        <!-- A few different tags are necessary to create a table. -->
        <table>
            <thead>
                <th>Ocean</th>
                <th>Average Depth</th>
                <th>Maximum Depth</th>
            </thead>
            <tbody>
                <tr>
                    <td>Pacific</td>
                    <td>4280 m</td>
                    <td>10911 m</td>
                </tr>
                <tr>
                    <td>Atlantic</td>
                    <td>3646 m</td>
                    <td>8486 m</td>
                </tr>
            </tbody>
        </table>
    </body>
<html>	
\end{lstlisting}
\end{tcolorbox}
%
\section{Django}
\subsection{basics}
	\subsubsection{html with django}
	Use \texttt{\{\{VAR\_html\}\}} for html variables.\\
	Use \texttt{\{\% PYTHON\_COMMANDS\%\}} for python commands. Remember to end a loop with \texttt{\{\% end<LOOP\_NAME>\%\}}.
	\subsubsection{important packages}
	\ttfamily
	django.urls:
	\begin{itemize}
		\item include
		\item path('path\_name', content the path renders, name="name\_of\_the\_path")
		
		\textnormal{
		\texttt{'path\_name'}:
		The default option is \texttt{""}. It could also be a variable,for example, \texttt{"<str:VAR\_NAME>"}.
		}
	\end{itemize}
	django.shortcuts:
	\begin{itemize}
		\item render(request, "FILE\_NAME.html", \{"VAR\_html":VAR\_python\})
	\end{itemize}
	\subsection{basic setup}
	\begin{itemize}
	\ttfamily
		\item pip3 install Django
		\item dango-admin startproject PROJECT\hN
		\item python manage.py startapp APP\hN
	\end{itemize}
	\normalfont
	When creating a new app, add the app's name to the list of \texttt{INSTALLED\_APPS} in \textit{settings.py}. Also create a \textit{urls.py} file in the \textit{APP\hN} folder.
	
	To direct the url to the desired app, \texttt{include("APP\hN.urls")} in the \texttt{path 'APP\hN/'}. Put the \texttt{path} in the list of \texttt{urlpatterns}.		
	\subsection{routes}
		\begin{itemize}
		\item kk
	\end{itemize}

\end{document}