\documentclass[12,a4paper]{article}
\newcommand{\hN}{\_NAME}
%
\title{Web design notes}
\author{haozepangthegoat}
%
\usepackage{listings}
\lstset{
  %language         = C++,
  basicstyle       = \ttfamily \color{VSwhite},
  keywordstyle     = \color{VSblue}\textbf,
  %keywordstyle     = \color{VSblue1}\textbf
  commentstyle     = \color{gray},
  stringstyle      = \color{VSwhite},
  stringstyle      = \color{VSOrange},
  columns          = fullflexible,
  numbers          = left,
  numberstyle      = \scriptsize\sffamily\color{gray},
  %caption          = A hello world program in \Cpp,
  xleftmargin      = 0.16\textwidth,
  xrightmargin     = 0.16\textwidth,
  showstringspaces = false,
  float,
}
\usepackage{tcolorbox}
%
\usepackage{xcolor}
\definecolor{VSblue}{RGB}{102, 152, 206}
\definecolor{VSblue1}{RGB}{157, 201, 230}
\definecolor{VSOrange}{RGB}{197, 148 124}
\definecolor{VSblack}{RGB}{31,31,31}
\definecolor{VSwhite}{RGB}{200,200 200}
%
\begin{document}
\maketitle
\section{Git}
\ttfamily
commands:
\begin{itemize}
		\item mkdir testfolder
		\item git clone REPOSITORY\_URL
		\item touch NEW\_FILE\_NAME
		\item git add FILE\_NAME
		\item git add . 
		\item git commit -m "messages"
		\item git push
		\item git commit -am "messages"
		\item git pull
		\item git branch
		\item git checkout -b NEW\_BRANCH
		\item git checkout BRANCH\_NAME\\
	
		\color{blue}
		\textbf{Advanced}
		\item git log
		\item git revert --hard COMMIT\_HASH
		\item git push --set-upstream origin colour\_box
\end{itemize}
\normalfont
To exclude files from git, create a file \texttt{.gitignore} and add the file intended to exclude to \texttt{.gitignore}. Use \texttt{*} as wildword. 

%S
\section{HTML}
%box1
\begin{tcolorbox}[title= \centering Example code, colback= VSblack, width =150mm]
\centering
	\begin{lstlisting}[language = HTML]

<!DOCTYPE html>
<html lang="en">
    <head>
        <title>New Year Check</title>
        <link href="" rel="stylesheet">
    </head>
    <body>
        
            <h1>Yes</h1>
        
            <h1>No</h1>
        
    </body>
</html>	\end{lstlisting}
\end{tcolorbox}
%
\section{Django}
\subsection{basics}
	\subsubsection{html with django}
	Use \texttt{\{\{VAR\_html\}\}} for html variables.\\
	Use \texttt{\{\% PYTHON\_COMMANDS\%\}} for python commands. Remember to end a loop with \texttt{\{\% end<LOOP\_NAME>\%\}}.
	\subsubsection{important packages}
	\ttfamily
	django.urls:
	\begin{itemize}
		\item include
		\item path('path\_name', content the path renders, name="name\_of\_the\_path")
		
		\textnormal{
		\texttt{'path\_name'}:
		The default option is \texttt{""}. It could also be a variable,for example, \texttt{"<str:VAR\_NAME>"}.
		}
	\end{itemize}
	django.shortcuts:
	\begin{itemize}
		\item render(request, "FILE\_NAME.html", \{"VAR\_html":VAR\_python\})
	\end{itemize}
	\subsection{basic setup}
	\begin{itemize}
	\ttfamily
		\item pip3 install Django
		\item dango-admin startproject PROJECT\hN
		\item python manage.py startapp APP\hN
	\end{itemize}
	\normalfont
	When creating a new app, add the app's name to the list of \texttt{INSTALLED\_APPS} in \textit{settings.py}. Also create a \textit{urls.py} file in the \textit{APP\hN} folder.
	
	To direct the url to the desired app, \texttt{include("APP\hN.urls")} in the \texttt{path 'APP\hN/'}. Put the \texttt{path} in the list of \texttt{urlpatterns}.		
	\subsection{routes}
		\begin{itemize}
		\item kk
	\end{itemize}

\end{document}